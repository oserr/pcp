\documentclass[11pt]{article}

\usepackage{amsmath}
\usepackage{amssymb}
\usepackage{amsfonts}
\usepackage{amsthm}
\usepackage{backnaur}
\usepackage[scaled]{beramono}
\usepackage{bm}
\usepackage[small,bf]{caption}
\usepackage[strict]{changepage}
\usepackage{cite}
\usepackage{dblfloatfix}
\usepackage{enumerate}
\usepackage{enumitem}
\usepackage{flushend}
\usepackage[T1]{fontenc}
\usepackage{graphicx}
\usepackage{ifsym}
\usepackage{lipsum}
\usepackage{listings}
\usepackage{makeidx}
\usepackage{mathrsfs}
\usepackage{multirow}
\usepackage{pdfpages}
\usepackage{subcaption}
\usepackage{setspace}
\usepackage{textcomp}
\usepackage[hyphens]{url}
\usepackage{booktabs}
\usepackage{multirow}
\usepackage{xcolor}
\usepackage{pgfgantt}
\usepackage{wrapfig}
\usepackage{balance}
\usepackage{tikz}
\usetikzlibrary{shapes,decorations}
\usepackage{pgfplots}
\usepgfplotslibrary{units}
\pgfplotsset{compat=1.14}
\usepackage{bm}
\usepackage{hyperref}
\hypersetup{
    colorlinks=true,
    linkcolor=blue,
    filecolor=magenta,
    urlcolor=cyan,
}

\addtolength{\evensidemargin}{-.5in}
\addtolength{\oddsidemargin}{-.5in}
\addtolength{\textwidth}{0.8in}
\addtolength{\textheight}{0.8in}
\addtolength{\topmargin}{-.4in}
\newcommand{\hw}{4}
%%%%%%%%%%%%%%%%%%%%%%%%%%%%%%
%%%%%%%%%%%%%%%%%%%%%%%%%%%%%%
%%%%%%%%%%%%%%%%%%%%%%%%%%%%%%
\title{\vspace{-25pt}
\huge CS 15-618\hfill Project Proposal}
\author{
    Patricio Chilano (pchilano) \\
    Omar Serrano (oserrano)
}
\date{\today}

\begin{document}

\definecolor{beaublue}{rgb}{0.74, 0.83, 0.9}

\lstset{
    language=C++,
    basicstyle=\ttfamily\scriptsize,
    keywordstyle=\color{blue}\ttfamily,
    stringstyle=\color{red}\ttfamily,
    commentstyle=\color{orange}\ttfamily,
    morecomment=[l][\color{magenta}]{\#},
    breaklines=true
}

\maketitle

\section*{Project Ideas}

We haven't decided on a project idea yet, but have an idea that we want to
explore in more depth this week before making a decision. Our goal is to make a
decision and commit to a project idea by Friday, April 6. Our project ideas are:

\begin{enumerate}
\item Parallelizing \texttt{libtomcrypt}.
\item Having more fun with GraphRats!
\end{enumerate}

We haven't yet commited ourselves to one of these two ideas, but our focus for
this week is to become more familiar with \texttt{libtomcrypt} to determine if
making it parallel is viable. We like GraphRats, but don't like the fact that
it's not a practical application and hence would be awkward to put the project
in our resumes, and are therefore treating it like a backup idea.

\section*{Parallel libtomcrypt}

\href{https://github.com/libtom/libtomcrypt}{libtomcrypt} is a popular open
source cryptographic library, and our goal would be to parallelize one or more
encryption algorithms in the library.

% What's interesting about it?
This project would be interesting because encryption is a prevalent component in
many computer systems and applications, so there is practical benefit to
speeding up these algorithms. For example, many online services rely on SSL/TLS
for secure communication, and these rely on encryption algorithms. Thus,
speeding up the encryption algorithms can lead to lower-latency service.
Furthermore, it would be particularly satisfying to be able to apply the
concepts we've learned in class to speed-up a popular library that is used in
practice.

% Why is it a suitable project?
One key aspect about this idea that makes it suitable to be a project is the
fact that the performance of the library can be used as a base benchmark,
allowing us to measure our performance by simply measuring the speedup. However,
there are many encryption libraries, so we could also compare the performance of
the algorithms we modify with those of other libraries. Furthermore, encryption
algorithms are not embarrassignly parallel, as can be seen in the small snippted
of code below, which is part of the logic in the AES encryption algorithm, with
clear dependencies between the variables inside the \texttt{for} loop. Parallelizing
these encryption algorithms is very likely not going to be trivial.

\begin{lstlisting}
for (;;) {
    t0 = Te0(byte(s0, 3)) ^ Te1(byte(s1, 2)) ^ Te2(byte(s2, 1)) ^ Te3(byte(s3, 0)) ^ rk[4];
    t1 = Te0(byte(s1, 3)) ^ Te1(byte(s2, 2)) ^ Te2(byte(s3, 1)) ^ Te3(byte(s0, 0)) ^ rk[5];
    t2 = Te0(byte(s2, 3)) ^ Te1(byte(s3, 2)) ^ Te2(byte(s0, 1)) ^ Te3(byte(s1, 0)) ^ rk[6];
    t3 = Te0(byte(s3, 3)) ^ Te1(byte(s0, 2)) ^ Te2(byte(s1, 1)) ^ Te3(byte(s2, 0)) ^ rk[7];

    rk += 8;
    if (--r == 0) {
        break;
    }

    s0 = Te0(byte(t0, 3)) ^ Te1(byte(t1, 2)) ^ Te2(byte(t2, 1)) ^ Te3(byte(t3, 0)) ^ rk[0];
    s1 = Te0(byte(t1, 3)) ^ Te1(byte(t2, 2)) ^ Te2(byte(t3, 1)) ^ Te3(byte(t0, 0)) ^ rk[1];
    s2 = Te0(byte(t2, 3)) ^ Te1(byte(t3, 2)) ^ Te2(byte(t0, 1)) ^ Te3(byte(t1, 0)) ^ rk[2];
    s3 = Te0(byte(t3, 3)) ^ Te1(byte(t0, 2)) ^ Te2(byte(t1, 1)) ^ Te3(byte(t2, 0)) ^ rk[3];
}
\end{lstlisting}

% What we need to resolve before deciding
Before committing ourselves to this project, we want to determine that there are
one or more encryption algorithms amenable to parallelization, and that we can
familiarize ourselves with the code well enough quickly in order to be able to
focus on parallelizing the algorithms rather than simply understanding what is
happening in the code.

Note that we may decide to proceed with the topic of encryption, but to do so
with another library, or to simply create a small library of a few parallel
encryption algorithms that we can then compare with another library.

% Any literatur consulted?

\section*{More fun with GraphRats!}

We like GraphRats because there are a lot of different parallelization avenues
we can explore. We almost certainly would map GraphRats to the GPU, but we would
also be interested in doing any of the following:

\begin{itemize}
\item
Map GraphRats to the Xeon Phi cores.
\item
Map GraphRats to ISPC.
\item
Use MPI on more than one machine, in combination with multi-threading, with the
possibility of exploring performance on larger workloads (i.e., larger graphs
with more rats).
\item Extreme speedup: combine MPI, multi-threading, and vector instructions.
\end{itemize}

% What's interesting about it?
% What we need to resolve before deciding
% Any literatur consulted?
% Why is it a suitable project?
% Any literatur consulted?
One of the benefits of working on GraphRats is that we are already familiar with
the code, so we can jump directly into mapping the code to GPU, or to exploit
some other parallelization technique. Furthermore, we would be able to dive
deeper into why some changes work and others don't, allowing us to produce a
final report with more empirical analysis.

\end{document}
