\documentclass[11pt]{article}

\usepackage{amsmath}
\usepackage{amssymb}
\usepackage{amsfonts}
\usepackage{amsthm}
\usepackage{backnaur}
\usepackage[scaled]{beramono}
\usepackage{bm}
\usepackage[small,bf]{caption}
\usepackage[strict]{changepage}
\usepackage{dblfloatfix}
\usepackage{enumerate}
\usepackage{enumitem}
\usepackage{flushend}
\usepackage[T1]{fontenc}
\usepackage{graphicx}
\usepackage{ifsym}
\usepackage{lipsum}
\usepackage{listings}
\usepackage{makeidx}
\usepackage{mathrsfs}
\usepackage{multirow}
\usepackage{pdfpages}
\usepackage{subcaption}
\usepackage{setspace}
\usepackage{textcomp}
\usepackage[hyphens]{url}
\usepackage{booktabs}
\usepackage{multirow}
\usepackage{xcolor}
\usepackage{pgfgantt}
\usepackage{wrapfig}
\usepackage{balance}
\usepackage{tikz}
\usetikzlibrary{shapes,decorations}
\usepackage{pgfplots}
\usepgfplotslibrary{units}
\pgfplotsset{compat=1.14}
\usepackage{bm}
\usepackage[
backend=biber,
style=ieee
]{biblatex}
\usepackage{hyperref}
\hypersetup{
    colorlinks=true,
    linkcolor=blue,
    filecolor=magenta,
    urlcolor=cyan,
}
%\bibliographystyle{IEEEtran}
%\bibliographystyle{acm}
\addbibresource{references.bib}

\addtolength{\evensidemargin}{-.5in}
\addtolength{\oddsidemargin}{-.5in}
\addtolength{\textwidth}{0.8in}
\addtolength{\textheight}{0.8in}
\addtolength{\topmargin}{-.4in}
%%%%%%%%%%%%%%%%%%%%%%%%%%%%%%
%%%%%%%%%%%%%%%%%%%%%%%%%%%%%%
%%%%%%%%%%%%%%%%%%%%%%%%%%%%%%
\title{\vspace{-25pt}
\huge CS 15-618 Project Proposal \\
\huge Synchrony
}
\author{
    Patricio Chilano (pchilano) \\
    Omar Serrano (oserrano)
}
\date{\today}

\begin{document}

\definecolor{beaublue}{rgb}{0.74, 0.83, 0.9}

\lstset{
    language=C++,
    basicstyle=\ttfamily\scriptsize,
    keywordstyle=\color{blue}\ttfamily,
    stringstyle=\color{red}\ttfamily,
    commentstyle=\color{orange}\ttfamily,
    morecomment=[l][\color{magenta}]{\#},
    breaklines=true
}

\maketitle

\section*{Summary}
We are going to implement a concurrent doubly-linked list, and a hash map, with
coarse-grained locks, fine-grained locks, and lock-free, in order to investigate
how these concurrent data structures are implemented, and to study first-hand
how these techniques compare with each other on a variety of different
workloads.

\section*{Background}

\section*{Challenge}

\section*{Resources}

\section*{Goals and Deliverables}

\begin{itemize}
\item Doubly linked list (DLL) with
\begin{itemize}
\item coarse-grained locks
\item fined-grained locks
\item lock-free
\end{itemize}
\item Test harness for linked list
\item Hash map (HM) with
\begin{itemize}
\item coarse-grained locks
\item fined-grained locks
\item lock-free
\end{itemize}
\item test harness for hash map
\end{itemize}

\section*{Platform Choice}

\section*{Schedule}

\printbibliography

\begin{table}[t]
\begin{center}
\begin{tabular}{ll}
\toprule
\bf Week & \bf Deliverable   \\
\midrule
4/9      & DLL with coarse grained locks \\
         & DLL Test harness \\
4/16     & Check point 1 \\
         & DLL fined-grained locks \\
         & HM coarse-grained locks \\
         & HM test-harness \\
4/23     & Check point 2 \\
         & DLL lock-free \\
         & HM fine-grained \\
4/30     & HM lock-free \\
         & Report \\
\bottomrule
\end{tabular}
\caption{Weekly Schedule}
\label{table:shedule}
\end{center}
\end{table}

\end{document}
