\documentclass[11pt]{article}

\usepackage{amsmath}
\usepackage{amssymb}
\usepackage{amsfonts}
\usepackage{amsthm}
\usepackage{backnaur}
\usepackage[scaled]{beramono}
\usepackage{bm}
\usepackage[small,bf]{caption}
\usepackage[strict]{changepage}
\usepackage{dblfloatfix}
\usepackage{enumerate}
\usepackage{enumitem}
\usepackage{flushend}
\usepackage[T1]{fontenc}
\usepackage{graphicx}
\usepackage{ifsym}
\usepackage{lipsum}
\usepackage{listings}
\usepackage{makeidx}
\usepackage{mathrsfs}
\usepackage{multirow}
\usepackage{pdfpages}
\usepackage{subcaption}
\usepackage{setspace}
\usepackage{textcomp}
\usepackage[hyphens]{url}
\usepackage{booktabs}
\usepackage{multirow}
\usepackage{xcolor}
\usepackage{pgfgantt}
\usepackage{wrapfig}
\usepackage{balance}
\usepackage{tikz}
\usetikzlibrary{shapes,decorations}
\usepackage{pgfplots}
\usepgfplotslibrary{units}
\pgfplotsset{compat=1.14}
\usepackage{bm}
\usepackage[
backend=biber,
style=ieee
]{biblatex}
\usepackage{hyperref}
\hypersetup{
    colorlinks=true,
    linkcolor=blue,
    filecolor=magenta,
    urlcolor=cyan,
}

\newcommand{\rt}{\textsuperscript{\textregistered}}
\newcommand{\tm}{\texttrademark}

\addtolength{\evensidemargin}{-.5in}
\addtolength{\oddsidemargin}{-.5in}
\addtolength{\textwidth}{0.8in}
\addtolength{\textheight}{0.8in}
\addtolength{\topmargin}{-.4in}
%%%%%%%%%%%%%%%%%%%%%%%%%%%%%%
%%%%%%%%%%%%%%%%%%%%%%%%%%%%%%
%%%%%%%%%%%%%%%%%%%%%%%%%%%%%%
\title{\vspace{-25pt}
\huge CS 15-618 Project Checkpoint 1 \\
\huge Synchrony
}
\author{
    Patricio Chilano (pchilano) \\
    Omar Serrano (oserrano)
}
\date{\today}

\begin{document}

\definecolor{beaublue}{rgb}{0.74, 0.83, 0.9}

\lstset{
    language=C++,
    basicstyle=\ttfamily\scriptsize,
    keywordstyle=\color{blue}\ttfamily,
    stringstyle=\color{red}\ttfamily,
    commentstyle=\color{orange}\ttfamily,
    morecomment=[l][\color{magenta}]{\#},
    breaklines=true,
    morekeywords={nullptr,noexcept}
}

\maketitle

\section*{Summary}
We are right on track with our original schedule, and have no concerns with
respect to our goals.

\begin{figure}
\begin{center}
%\begin{tabular}{c}
\begin{lstlisting}
/**
 * A simple doubly linked list with very basic operations:
 * - inserting at the front of the list
 * - removing anywhere from the list
 * - and finding elements
 */
template <typename T> struct DlList {
  DlNode<T> *head{nullptr};
  unsigned size{0};

  ~DlList();
  DlNode<T> *Insert(T value);
  bool Remove(T value);
  bool Contains(T value) const noexcept;
  unsigned Size() const noexcept;
  bool Empty() const noexcept;
};
\end{lstlisting}
\caption{
This is the single threaded list, but other lists share the same interface.}
\label{fig:dllist}
%\end{tabular}
\end{center}
\end{figure}

\section*{Work Done}
We setup our git repo and created a couple scripts to setup our work
environment: one to setup our git repo to use git-clang-format, and another one
to install a newer version of cmake on the GHC machines. We also created a basic
single-threaded doubly-linked list, a coarse-grained multi-threaded
doubly-linked list, and a hash map that can be configured with different linked
lists to be used as buckets. We also used GoogleTest to create unit tests for
the lists and the hash map. Figure~\ref{fig:dllist} shows the interface for our
lists.

\begin{table}[t]
\begin{center}
\begin{tabular}{lll}
\toprule
\bf Week & \bf Assigned To  & \bf Deliverable   \\
\midrule
4/16     & O,P   & Doubly-linked list with fine-grain locks \\
         & P     & Hash map with fine-grain locks \\
         & O     & Benchmarking harness for linked-lists \\
         & O     & Regression unit tests for hash map \\
4/23     & P     & Benchmarking harness for hash maps \\
         & P     & Benchmarks for linked lists with coarse- and fine-grain locks \\
         & O     & Benchmarks for hash maps with coarse- and fine-grain locks \\
         & O,P   & {\bf Check point 2} \\
         & O,P   & Lock-free doubly-linked list \\
         & O     & Benchmarks for Java ConcurrentLinkedDeque \\
4/30     & P     & Benchmarks for Java ConcurrentHashMap \\
         & O,P   & Lock-free doubly-linked list \\
         & O,P   & Lock-free hash map \\
         & P     & Benchmark for lock-free linked list \\
         & O     & Benchmark for lock-free hash map \\
         & O,P   & Final report \\
         & *     & *Lock-free ordered doubly-linked list \\
         & *     & *Concurrent data structures with spinning locks \\
\bottomrule
\end{tabular}
\caption{
Weekly schedule. Check points include the items preceding them. The * on the
last two items indicate {\it nice-to-haves}, but we are not commiting to them.
The {\it Assigned To} column indicates with an O for Omar or a P for Patricio
who will work on a given task.
}
\label{table:sche}
\end{center}
\end{table}

\section*{Goals}
We are still planning to deliver the following:
\begin{itemize}
\item
Implement a concurrent doubly linked list using different
synchronization methods: course-grain locks, fine-grain locks and lock-free
techniques.
\item Compare the performance of each implementation of the doubly
linked list against each other and against the serial version with respect to
the number of threads.
\item Use the doubly linked list as a dequeue for each implementation and compare
their performance against the Java ConcurrentLinkedDeque data structure.
\item Implement a concurrent HashMap using different synchronization methods:
course-grain locks, fine-grain locks and lock-free techniques.
\item Compare the performance of each implementation of the HashMap against
each other and against the serial version with respect to the number of threads.
\item Compare the performance of each implementation of the HashMap against
the Java ConcurrentHashMap data structure.
\end{itemize}

We are still including a couple {\it nice-to-haves} in our schedule, per
Table~\ref{table:sche}, but it seems unlikely that we'll get to them.

\section*{Schedule}
Table~\ref{table:sche} contains an updated version of the schedule. The new
schedule is the same as the original, minus completed tasks, all of which we
expected to completed by this checkpoint. The table contains a column to
indicate how tasks are assigned. We are currently right on track with our
original schedule.

\end{document}
