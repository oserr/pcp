\documentclass[11pt]{article}

\usepackage{amsmath}
\usepackage{amssymb}
\usepackage{amsfonts}
\usepackage{amsthm}
\usepackage{backnaur}
\usepackage[scaled]{beramono}
\usepackage{bm}
\usepackage[small,bf]{caption}
\usepackage[strict]{changepage}
\usepackage{dblfloatfix}
\usepackage{enumerate}
\usepackage{enumitem}
\usepackage{flushend}
\usepackage[T1]{fontenc}
\usepackage{graphicx}
\usepackage{ifsym}
\usepackage{lipsum}
\usepackage{listings}
\usepackage{makeidx}
\usepackage{mathrsfs}
\usepackage{multirow}
\usepackage{pdfpages}
\usepackage{subcaption}
\usepackage{setspace}
\usepackage{textcomp}
\usepackage[hyphens]{url}
\usepackage{booktabs}
\usepackage{multirow}
\usepackage{xcolor}
\usepackage{pgfgantt}
\usepackage{wrapfig}
\usepackage{balance}
\usepackage{tikz}
\usetikzlibrary{shapes,decorations}
\usepackage{pgfplots}
\usepgfplotslibrary{units}
\pgfplotsset{compat=1.14}
\usepackage{bm}
\usepackage[
backend=biber,
style=ieee
]{biblatex}
\usepackage{hyperref}
\hypersetup{
    colorlinks=true,
    linkcolor=blue,
    filecolor=magenta,
    urlcolor=cyan,
}

\newcommand{\rt}{\textsuperscript{\textregistered}}
\newcommand{\tm}{\texttrademark}

\addtolength{\evensidemargin}{-.5in}
\addtolength{\oddsidemargin}{-.5in}
\addtolength{\textwidth}{0.8in}
\addtolength{\textheight}{0.8in}
\addtolength{\topmargin}{-.4in}
%%%%%%%%%%%%%%%%%%%%%%%%%%%%%%
%%%%%%%%%%%%%%%%%%%%%%%%%%%%%%
%%%%%%%%%%%%%%%%%%%%%%%%%%%%%%
\title{\vspace{-25pt}
\huge CS 15-618 Project Checkpoint 1 \\
\huge Synchrony (ID 24)
}
\author{
    Patricio Chilano (pchilano) \\
    Omar Serrano (oserrano)
}
\date{\today}

\begin{document}

\definecolor{beaublue}{rgb}{0.74, 0.83, 0.9}

\lstset{
    language=C++,
    basicstyle=\ttfamily\scriptsize,
    keywordstyle=\color{blue}\ttfamily,
    stringstyle=\color{red}\ttfamily,
    commentstyle=\color{orange}\ttfamily,
    morecomment=[l][\color{magenta}]{\#},
    breaklines=true,
    morekeywords={nullptr,noexcept}
}

\maketitle

\section*{Summary}
We are right on track with our original schedule, but we are making some updates to 
our plans in response to the feedback we received on our proposal, which includes:

\begin{itemize}
\item
Reframing our approach to the problem to be more purposeful.
\item
Substitute the Java concurrent package with a C/C++ multi-threading library when making comparisons with external libraries.
\item
Change our {\it nice-to-have} to include parallelization of a problem or algorithm using
our concurrent data structures.
\end{itemize}

All of these are in a fluid state because we've just received the feedback, but
these pivots shouldn't affect us negatively, because we are still working on
tasks that are not affected by them.

\begin{figure}
\begin{center}
%\begin{tabular}{c}
\begin{lstlisting}
/**
 * A simple doubly linked list with basic operations:
 */
template <typename T> struct DlList {
  DlNode<T> *head{nullptr};
  unsigned size{0};

  ~DlList();
  DlNode<T> *Insert(T value);
  bool Remove(T value);
  bool Contains(T value) const noexcept;
  T *Find(T value) const noexcept;
  unsigned Size() const noexcept;
  bool Empty() const noexcept;
};
\end{lstlisting}
\caption{
This is the single threaded list, but other lists share the same interface.}
\label{fig:dllist}
%\end{tabular}
\end{center}
\end{figure}

\begin{figure}
\begin{center}
%\begin{tabular}{c}
\begin{lstlisting}
/**
 * A simple HashMap with basic operations:
*/
template <typename K, typename V> class HashMap {
  class Element {
    K key;
    V value;
  };
  DlList<Element> map[MAX_BUCKETS];
  unsigned size{0};

  void Insert(K key, V value);
  void Remove(K key);
  bool Has(K key);
  unsigned Size() const noexcept;
  // using subscripts to read or write to the map
  V &operator[](K key);  
};
\end{lstlisting}
\caption{
This is the HashMap class. It can be easily changed to use any particular list for the buckets}
\label{fig:hashmap}
%\end{tabular}
\end{center}
\end{figure}

\section*{Work Done}
We created our git repo along with a couple of scripts to setup our working
environment: one script to setup git to use git-clang-format so that uniform
formatting is maintained among teammates, and another one to install and setup the cmake environment used for
the building process. 

We implemented a basic single-threaded doubly-linked list and a coarse-grained multi-threaded
doubly-linked list. We also created a HashMap class that can be changed to use any 
specific doubly-linked list for implementing the buckets. Since each bucket can be managed 
independently from the others, once we have a doubly-linked list implemented with some specific 
synchronization mechanism, implementing the equivalent for the HashMap should be straighforward.  Figure~\ref{fig:dllist} and Figure~\ref{fig:hashmap} show the interface for our list and HashMap respectively.

Finally we used GoogleTest to create unit tests for the lists and the HashMap. As of now we just created
tests to check for correctness, not benchmarks to measure speedup.

\section*{Goals}
We are still planning to deliver the following:
\begin{itemize}
\item
Implement a concurrent doubly linked list using different
synchronization methods: course-grain locks, fine-grain locks and lock-free
techniques.
\item Compare the performance of each implementation of the doubly
linked list against each other and against the serial version with respect to
the number of threads.
\item Implement a concurrent HashMap using different synchronization methods:
course-grain locks, fine-grain locks and lock-free techniques.
\item Compare the performance of each implementation of the HashMap against
each other and against the serial version with respect to the number of threads.
\item Use some external C/C++ library that implements concurrent lists and
HashMaps to compare it against our versions.
\end{itemize}

Instead of trying another data structure as a {\it nice-to-have}, we are now
looking forward to the opportunity of applying our concurrent data structures to
an algorithm or problem. We don't have concrete ideas right now, but have
floated the idea of doing something like a search or graph algorithm, like
$A^*$, for example.

\begin{table}[t]
\begin{center}
\begin{tabular}{lll}
\toprule
\bf Week & \bf Assigned To  & \bf Deliverable   \\
\midrule
4/16     & O     & Doubly-linked list with fine-grain locks \\
         & O,P   & Find third party C/C++ multi-threading library \\
         & P     & Lock-free doubly-linked list \\
4/23     & O     & Hash map with fine-grain locks \\
         & P     & Lock-free HashMap \\ 
         & O,P   & {\bf Check point 2} \\   
4/30     & O     & Benchmarking harness for linked-lists \\
         & P     & Benchmarking harness for HashMap \\
         & P,O   & Benchmarks for third party C/C++ multi-threading library \\
         & P,O   & Compare results with hypothesis. Identify improvement spots \\
         & O,P   & Final report \\
         & *     & *Parallelize algorithm/problem with our data structures \\
\bottomrule
\end{tabular}
\caption{
Weekly schedule. Check points include the items preceding them. The * on the
last means it is a {\it nice-to-have}, but we are not commiting to it.
The {\it Assigned To} column indicates with an O for Omar or a P for Patricio
who will work on a given task.
}
\label{table:sche}
\end{center}
\end{table}

\section*{Schedule}
Table~\ref{table:sche} contains an updated version of the schedule. The new
schedule is the same as the original, minus completed tasks, and three
modifications:

\begin{itemize}
\item
Remove Java-related tasks, since we won't be using Java's concurrent package.
\item
Add a task for this week to find a C/C++ multi-threading library to use for
benchmarking.
\item
Change {\it nice-to-have} to one where we apply our concurrent data structures
to an algorithm or a problem.
\end{itemize}

The table contains a column to indicate how tasks are assigned. We are currently
right on track with our original schedule.

\section*{Pending Issues}
This is a summary of the pending issues we have to resolve:
\begin{itemize}
\item
Define the concrete techniques we are going to use to implement fine-grain locking and lock-free mechanisms.
\item
Find a C/C++ library to use for benchmarking.
\item
Think of a problem or algorithm that could be parallelized using our data structures.
\end{itemize}

\end{document}
