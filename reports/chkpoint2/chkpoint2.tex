\documentclass[11pt]{article}

\usepackage{amsmath}
\usepackage{amssymb}
\usepackage{amsfonts}
\usepackage{amsthm}
\usepackage{backnaur}
\usepackage[scaled]{beramono}
\usepackage{bm}
\usepackage[small,bf]{caption}
\usepackage[strict]{changepage}
\usepackage{dblfloatfix}
\usepackage{enumerate}
\usepackage{enumitem}
\usepackage{flushend}
\usepackage[T1]{fontenc}
\usepackage{graphicx}
\usepackage{ifsym}
\usepackage{lipsum}
\usepackage{listings}
\usepackage{makeidx}
\usepackage{mathrsfs}
\usepackage{multirow}
\usepackage{pdfpages}
\usepackage{subcaption}
\usepackage{setspace}
\usepackage{textcomp}
\usepackage[hyphens]{url}
\usepackage{booktabs}
\usepackage{multirow}
\usepackage{xcolor}
\usepackage{pgfgantt}
\usepackage{wrapfig}
\usepackage{balance}
\usepackage{tikz}
\usetikzlibrary{shapes,decorations}
\usepackage{pgfplots}
\usepgfplotslibrary{units}
\pgfplotsset{compat=1.14}
\usepackage{bm}
\usepackage[
backend=biber,
style=ieee
]{biblatex}
\usepackage{hyperref}
\hypersetup{
    colorlinks=true,
    linkcolor=blue,
    filecolor=magenta,
    urlcolor=cyan,
}
\addbibresource{references.bib}

\newcommand{\rt}{\textsuperscript{\textregistered}}
\newcommand{\tm}{\texttrademark}

\addtolength{\evensidemargin}{-.5in}
\addtolength{\oddsidemargin}{-.5in}
\addtolength{\textwidth}{0.8in}
\addtolength{\textheight}{0.8in}
\addtolength{\topmargin}{-.4in}
%%%%%%%%%%%%%%%%%%%%%%%%%%%%%%
%%%%%%%%%%%%%%%%%%%%%%%%%%%%%%
%%%%%%%%%%%%%%%%%%%%%%%%%%%%%%
\title{\vspace{-25pt}
\huge CS 15-618 Project Checkpoint 2 \\
\huge Synchrony (ID 24)
}
\author{
    Patricio Chilano (pchilano) \\
    Omar Serrano (oserrano)
}
\date{\today}

\begin{document}

\definecolor{beaublue}{rgb}{0.74, 0.83, 0.9}

\lstset{
    language=C++,
    basicstyle=\ttfamily\scriptsize,
    keywordstyle=\color{blue}\ttfamily,
    stringstyle=\color{red}\ttfamily,
    commentstyle=\color{orange}\ttfamily,
    morecomment=[l][\color{magenta}]{\#},
    breaklines=true,
    morekeywords={nullptr,noexcept}
}

\maketitle

\section*{Summary}
We are on track with our original schedule but still working on the
implementation of lock free linked lists that we had planned to have finished by
this checkpoint. We made a change in one of our data structures: we will use
single linked lists instead of doubly linked lists. From the papers we found on
lock free data structures, very few implement doubly linked lists and the ones
that do \cite{Sundell} are difficult and tricky to implement while still
guaranteeing correctness. Since the focus of this project is to have experience
implementing and comparing syncronization techniques we think it is more valuable
to use lock free single linked lists, which will also leave more time for actual
benchmarking and analysis. We've also begun making progress on the benchmark
harness for the linked-lists, and decided to add a linked list with fine-grained
spinning-reader-writer locks to the mix of lists we will use in our benchmarks.

\section*{Work Done}
We implemented a fine-grained singly-linked list. We used the hand-over-hand
locking technique presented in the lecture 17 slides. The code in the slides
contains some unnecessary checks of {\tt curr}, but overall the same
hand-over-hand pattern can be used in any function that traverses the list. We
noticed that the overall structure of the algorithms could be used with
different kinds of locks, so we decided to implement a spinning-reader-writer
lock to create a fine-grained spinning-reader-writer lock singly-linked list,
which we will also use in our benchmarks. The full implementation of the
spinning-reader-write lock is included in Figure~\ref{fig:rwlock}.

\begin{figure}
\begin{center}
\begin{lstlisting}
struct RwLock {
  std::atomic_int counter{0};

  void ReadLock() noexcept {
    int val, old;
    do {
      while ((val = counter.load()) < 0)
        ;
      old = val++;
    } while (not counter.compare_exchange_weak(old, val));
  }

  void ReadUnlock() noexcept { --counter; }

  void WriteLock() noexcept {
    int val, old;
    do {
      while ((val = counter.load()) != 0)
        ;
      old = val--;
    } while (not counter.compare_exchange_weak(old, val));
  }

  void WriteUnlock() noexcept { ++counter; }
};
\end{lstlisting}
\caption{Spinning-writer-reader lock implemented with atomics.}
\label{fig:rwlock}
\end{center}
\end{figure}

We also created more tests with GoogleTest. We converted the hashmap and list
unit tests to type-parametrized tests. This allows us to use the same set of
tests on any list or hashmap that supports the APIs we have defined. Instead of
creating different tests, we simply add one line with the type of list or
hashmap to be tested. These are serial tests that simply test that all the
functions work correctly on a single-threaded environment, but we also created a
type-parametrized multi-threaded concurrent list test that ensures lists work
correctly with multiple threads making insertions and removals simultaneously.

We began implementing the benchmark harness for the lists, which will allow us
to test a list with different parameters, including different number of threads,
total items to insert/remove from the list, and different use-profiles, e.g.,
80\% inserts, 10\% removes, and 10\% lookups vs 50\% inserts, 25\% removes, and
25\% lookups. What remains is to add logic to pass in all the parameters via the
command line, and perhaps to add additional logic for different kinds of tests,
e.g., preload list with items to be able to run a use-profile with 50\% removals
and 50\% lookups.

As for lock free data structures, we did research and found some papers that gave
us ideas on how to implement them. We will adopt a scheme based mainly on
\cite{Harris}, where we will use flags within pointer addresses to keep track of
deleted nodes. The delete operation will then be divided in a logical delete and
a physical delete requiring the use of the simple CAS atomic primitive available
in the x86-64 architecture. Also, we will use atomic types provided by C++ to
ensure sequential consistency of the code and avoid races.

We've also trimmed down our list of candidates for a C++ concurrent library to one
of the following:

\begin{itemize}
\item
\href{https://github.com/01org/tbb}{tbb} Intel's concurrent library, Threading
Building Blocks (TBB).
\item
\href{https://github.com/facebook/folly}{folly} Facebook's open-source library,
which includes concurrent data structures.
\item
\href{https://github.com/efficient/libcuckoo}{libcuckoo} library, another C++
library with concurrent hash tables.
\end{itemize}

TBB seems ideal, but we are not yet commited to it because we don't know how easy
it will be to build it and use it. As of now, the only thing we can say is that it
builds on Linux with a simple {\tt make}, but the same command on macOS will make
it barf (at least on Omar's mac).

\section*{Goals}
We are not removing any of our goals, but are modifying two of them. One change
is to make the lock-free list singly-linked, and the other change is to include
benchmarks for a fine-grained spinning-reader-writer-lock list. Our deliverables
are:

\begin{itemize}
\item
Implement a concurrent singly-linked list with different kinds of locks:
coarse-grained locks, fine-grained locks, and fine-grained
spinning-reader-writer locks.
\item
Implement a concurrent lock-free singly-linked list.
\item
Compare the performance of the lists, including serial version, with respect to
number of threads and use-profile, e.g., 80\% inserts, 10\% removals, 10\%
lookups.
\item
Implement concurrent hashmaps with each one of our lists.
\item
Compare the performance of the hashmaps, including serial version, with respect
to the number of threads and different use-profiles.
\item
Compare performance of our library with another C++ library, e.g., TBB.
\item
{\bf Nice-to-have}. Parallelize algorithm/problem with our data structures.
\end{itemize}

We will gather the results from these experiments into a series of graphs and
present them in the poster session. We are confident that we can complete all
our delivarables, but it seems highly unlikely that we'll have a chance to apply
our data structures to an algorithm or problem; however, if we are able to make
quick progress on our remaining tasks, then we'll definitely look into it, but
as of now we have not had any time to look into this.

\section*{Schedule}
Table~\ref{table:sche} contains an updated version of the schedule. The week is
packed with tasks, but we should be able to move through some of these
relatively quickly. The benchmarking harness for the lists is almost complete,
and this will make it easier for us to write the harness for the hashmaps and
the data structures for the external library, because at the very least we
should be able to use the list benchmark harness as a template. It's hard to
gage how long it will take us to complete some of the other tasks, such as
implementing the lock-free list, or getting the library to work, but ultimately
we believe that 9 days are enough to get us to the end.

\begin{table}[t]
\begin{center}
\begin{tabular}{lll}
\toprule
\bf Week & \bf Assigned To  & \bf Deliverable   \\
\midrule
4/30     & O     & Benchmarking harness for linked-lists \\
         & P     & Lock-free singly-linked list \\
         & O     & Benchmarking harness for HashMap \\
5/03     & P,O   & Benchmarks for third party C/C++ multi-threading library \\
         & P,O   & Compare results with hypothesis. Identify improvement spots \\
         & O,P   & Final report \\
         & O,P   & Poster \\
         & *     & *Parallelize algorithm/problem with our data structures \\
\bottomrule
\end{tabular}
\caption{
Schedule for last week. The {\it nice-to-have} is still in our list, but it
seems highly unlikely that we'll be able to get to it.
}
\label{table:sche}
\end{center}
\end{table}

\section*{Pending Issues}
Our issues are for the most part resolved, and we have a very good idea what
direction to take:

\begin{itemize}
\item
{\bf C++ concurrent library}. This is still in a somewhat fluid state, because
we've yet to benchmark a library, but we have at least three very good candidates.
We'll start our experimentation with {\tt tbb} and only move on to one of the
other libraries, or reach out for help, if we have trouble getting TBB to
cooperate.
\item
{\bf Parallelizing algorithm/problem}. This is our {\it nice-to-have} item and
hence don't see it as something that needs to be resolved, because it doesn't
affect our deliverables. If we are able to complete our remaining tasks quickly
then we'll shift our focus to this, and reach out for help if necessary.
\end{itemize}

\printbibliography

\end{document}
