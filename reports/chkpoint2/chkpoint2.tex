\documentclass[11pt]{article}

\usepackage{amsmath}
\usepackage{amssymb}
\usepackage{amsfonts}
\usepackage{amsthm}
\usepackage{backnaur}
\usepackage[scaled]{beramono}
\usepackage{bm}
\usepackage[small,bf]{caption}
\usepackage[strict]{changepage}
\usepackage{dblfloatfix}
\usepackage{enumerate}
\usepackage{enumitem}
\usepackage{flushend}
\usepackage[T1]{fontenc}
\usepackage{graphicx}
\usepackage{ifsym}
\usepackage{lipsum}
\usepackage{listings}
\usepackage{makeidx}
\usepackage{mathrsfs}
\usepackage{multirow}
\usepackage{pdfpages}
\usepackage{subcaption}
\usepackage{setspace}
\usepackage{textcomp}
\usepackage[hyphens]{url}
\usepackage{booktabs}
\usepackage{multirow}
\usepackage{xcolor}
\usepackage{pgfgantt}
\usepackage{wrapfig}
\usepackage{balance}
\usepackage{tikz}
\usetikzlibrary{shapes,decorations}
\usepackage{pgfplots}
\usepgfplotslibrary{units}
\pgfplotsset{compat=1.14}
\usepackage{bm}
\usepackage[
backend=biber,
style=ieee
]{biblatex}
\usepackage{hyperref}
\hypersetup{
    colorlinks=true,
    linkcolor=blue,
    filecolor=magenta,
    urlcolor=cyan,
}

\newcommand{\rt}{\textsuperscript{\textregistered}}
\newcommand{\tm}{\texttrademark}

\addtolength{\evensidemargin}{-.5in}
\addtolength{\oddsidemargin}{-.5in}
\addtolength{\textwidth}{0.8in}
\addtolength{\textheight}{0.8in}
\addtolength{\topmargin}{-.4in}
%%%%%%%%%%%%%%%%%%%%%%%%%%%%%%
%%%%%%%%%%%%%%%%%%%%%%%%%%%%%%
%%%%%%%%%%%%%%%%%%%%%%%%%%%%%%
\title{\vspace{-25pt}
\huge CS 15-618 Project Checkpoint 1 \\
\huge Synchrony (ID 24)
}
\author{
    Patricio Chilano (pchilano) \\
    Omar Serrano (oserrano)
}
\date{\today}

\begin{document}

\definecolor{beaublue}{rgb}{0.74, 0.83, 0.9}

\lstset{
    language=C++,
    basicstyle=\ttfamily\scriptsize,
    keywordstyle=\color{blue}\ttfamily,
    stringstyle=\color{red}\ttfamily,
    commentstyle=\color{orange}\ttfamily,
    morecomment=[l][\color{magenta}]{\#},
    breaklines=true,
    morekeywords={nullptr,noexcept}
}

\maketitle

\section*{Summary}
We are right on track with our original schedule but still working on the implementation
of lock free linked lists. We made a change in one of our data structures: we will use single linked
lists instead of doubly linked lists. From the papers we found on lock free data structures very 
few implement doubly linked lists (\cite{Sundell}) and they become difficult to implement
while guaranteeing correctness. Since the focus of this project is in having experience implementing 
and comparing syncronization techniques we think is more valuable to use single linked lists 
which will also leave for time for actual benchmarking and analysis.

\section*{Work Done}

As for lock free data structures we did research and found some papers that gave us ideas on how
to implement them. We will adopt a scheme based mainly on \cite{Harris}, where we will use flags within
pointer addresses to keep track of deleted nodes. This will also require to use the CAS atomic primitive
which is available in the x86-64 architecture that we are using as target. Also we will use atomic types
provided by C++ to ensure sequential consistency of the code and avoid races.

Finally regarding 

\section*{Goals}
We are planning to deliver the following:
\begin{itemize}
\item
Implement a concurrent single linked list using different
synchronization methods: course-grain locks, fine-grain locks and lock-free
techniques.
\item Compare the performance of each implementation of the single
linked list against each other and against the serial version with respect to
the number of threads.
\item Implement a concurrent HashMap using different synchronization methods:
course-grain locks, fine-grain locks and lock-free techniques.
\item Compare the performance of each implementation of the HashMap against
each other and against the serial version with respect to the number of threads.
\item Use external C/C++ library that implements concurrent lists and
HashMaps to compare it against our versions.
\end{itemize}

Instead of trying another data structure as a {\it nice-to-have}, we are now
looking forward to the opportunity of applying our concurrent data structures to
an algorithm or problem. We don't have concrete ideas right now, but have
floated the idea of doing something like a search or graph algorithm, like
$A^*$, for example.

\begin{table}[t]
\begin{center}
\begin{tabular}{lll}
\toprule
\bf Week & \bf Assigned To  & \bf Deliverable   \\
\midrule
4/16     & O     & Doubly-linked list with fine-grain locks \\
         & O,P   & Find third party C/C++ multi-threading library \\
         & P     & Lock-free doubly-linked list \\
4/23     & O     & Hash map with fine-grain locks \\
         & P     & Lock-free HashMap \\ 
         & O,P   & {\bf Check point 2} \\   
4/30     & O     & Benchmarking harness for linked-lists \\
         & P     & Benchmarking harness for HashMap \\
         & P,O   & Benchmarks for third party C/C++ multi-threading library \\
         & P,O   & Compare results with hypothesis. Identify improvement spots \\
         & O,P   & Final report \\
         & *     & *Parallelize algorithm/problem with our data structures \\
\bottomrule
\end{tabular}
\caption{
Weekly schedule. Check points include the items preceding them. The * on the
last means it is a {\it nice-to-have}, but we are not commiting to it.
The {\it Assigned To} column indicates with an O for Omar or a P for Patricio
who will work on a given task.
}
\label{table:sche}
\end{center}
\end{table}

\section*{Schedule}
Table~\ref{table:sche} contains an updated version of the schedule. The new
schedule is the same as the original, minus completed tasks, and three
modifications:

\begin{itemize}
\item
Remove Java-related tasks, since we won't be using Java's concurrent package.
\item
Add a task for this week to find a C/C++ multi-threading library to use for
benchmarking.
\item
Change {\it nice-to-have} to one where we apply our concurrent data structures
to an algorithm or a problem.
\end{itemize}

The table contains a column to indicate how tasks are assigned. We are currently
right on track with our original schedule.

\section*{Pending Issues}
This is a summary of the pending issues we have to resolve:
\begin{itemize}
\item
Define the concrete techniques we are going to use to implement fine-grain locking and lock-free mechanisms.
\item
Find a C/C++ library to use for benchmarking.
\item
Think of a problem or algorithm that could be parallelized using our data structures.
\end{itemize}

\end{document}
